% ****************************************************************************************************
% classicthesis-config.tex 
% formerly known as loadpackages.sty, classicthesis-ldpkg.sty, and classicthesis-preamble.sty 
% Use it at the beginning of your ClassicThesis.tex, or as a LaTeX Preamble 
% in your ClassicThesis.{tex,lyx} with % ****************************************************************************************************
% classicthesis-config.tex 
% formerly known as loadpackages.sty, classicthesis-ldpkg.sty, and classicthesis-preamble.sty 
% Use it at the beginning of your ClassicThesis.tex, or as a LaTeX Preamble 
% in your ClassicThesis.{tex,lyx} with % ****************************************************************************************************
% classicthesis-config.tex 
% formerly known as loadpackages.sty, classicthesis-ldpkg.sty, and classicthesis-preamble.sty 
% Use it at the beginning of your ClassicThesis.tex, or as a LaTeX Preamble 
% in your ClassicThesis.{tex,lyx} with % ****************************************************************************************************
% classicthesis-config.tex 
% formerly known as loadpackages.sty, classicthesis-ldpkg.sty, and classicthesis-preamble.sty 
% Use it at the beginning of your ClassicThesis.tex, or as a LaTeX Preamble 
% in your ClassicThesis.{tex,lyx} with \input{classicthesis-config}
% ****************************************************************************************************  
% If you like the classicthesis, then I would appreciate a postcard. 
% My address can be found in the file ClassicThesis.pdf. A collection 
% of the postcards I received so far is available online at 
% http://postcards.miede.de
% ****************************************************************************************************

% ****************************************************************************************************
% 1. Configure classicthesis for your needs here, e.g., remove "drafting" below 
% in order to deactivate the time-stamp on the pages
% ****************************************************************************************************
\PassOptionsToPackage{eulerchapternumbers,listings,drafting,%
				 pdfspacing,%floatperchapter,%linedheaders,%
				 subfig,beramono,eulermath,parts}{classicthesis}										
% ********************************************************************
% Available options for classicthesis.sty 
% (see ClassicThesis.pdf for more information):
% drafting
% parts nochapters linedheaders
% eulerchapternumbers beramono eulermath pdfspacing minionprospacing
% tocaligned dottedtoc manychapters
% listings floatperchapter subfig
% ********************************************************************

% ********************************************************************
% Triggers for this config
% ******************************************************************** 
\usepackage{ifthen}
\newboolean{enable-backrefs} % enable backrefs in the bibliography
\setboolean{enable-backrefs}{false} % true false
% ****************************************************************************************************


% ****************************************************************************************************
% 2. Personal data and user ad-hoc commands
% ****************************************************************************************************
\newcommand{\myTitle}{A Classic Thesis Style\xspace}
\newcommand{\mySubtitle}{An Homage to The Elements of Typographic Style\xspace}
\newcommand{\myDegree}{Doktor-Ingenieur (Dr.-Ing.)\xspace}
\newcommand{\myName}{Andr\'e Miede\xspace}
\newcommand{\myProf}{Put name here\xspace}
\newcommand{\myOtherProf}{Put name here\xspace}
\newcommand{\mySupervisor}{Put name here\xspace}
\newcommand{\myFaculty}{Put data here\xspace}
\newcommand{\myDepartment}{Put data here\xspace}
\newcommand{\myUni}{Put data here\xspace}
\newcommand{\myLocation}{Darmstadt\xspace}
\newcommand{\myTime}{December 2011\xspace}
\newcommand{\myVersion}{version 4.0\xspace}

% ********************************************************************
% Setup, finetuning, and useful commands
% ********************************************************************
\newcounter{dummy} % necessary for correct hyperlinks (to index, bib, etc.)
\newlength{\abcd} % for ab..z string length calculation
\providecommand{\mLyX}{L\kern-.1667em\lower.25em\hbox{Y}\kern-.125emX\@}
\newcommand{\ie}{i.\,e.}
\newcommand{\Ie}{I.\,e.}
\newcommand{\eg}{e.\,g.}
\newcommand{\Eg}{E.\,g.} 
% ****************************************************************************************************


% ****************************************************************************************************
% 3. Loading some handy packages
% ****************************************************************************************************
% ******************************************************************** 
% Packages with options that might require adjustments
% ******************************************************************** 
\PassOptionsToPackage{latin9}{inputenc}	% latin9 (ISO-8859-9) = latin1+"Euro sign"
 \usepackage{inputenc}				

%\PassOptionsToPackage{ngerman,american}{babel}   % change this to your language(s)
% Spanish languages need extra options in order to work with this template
%\PassOptionsToPackage{spanish,es-lcroman}{babel}
 \usepackage{babel}					

\PassOptionsToPackage{square,numbers}{natbib}
 \usepackage{natbib}				

\PassOptionsToPackage{fleqn}{amsmath}		% math environments and more by the AMS 
 \usepackage{amsmath}

% ******************************************************************** 
% General useful packages
% ******************************************************************** 
\PassOptionsToPackage{T1}{fontenc} % T2A for cyrillics
	\usepackage{fontenc}                 
\usepackage{xspace} % to get the spacing after macros right  
\usepackage{mparhack} % get marginpar right
\usepackage{fixltx2e} % fixes some LaTeX stuff 
\PassOptionsToPackage{printonlyused,smaller}{acronym}
	\usepackage{acronym} % nice macros for handling all acronyms in the thesis
%\renewcommand*{\acsfont}[1]{\textssc{#1}} % for MinionPro
\renewcommand{\bflabel}[1]{{#1}\hfill} % fix the list of acronyms
% ****************************************************************************************************


% ****************************************************************************************************
% 4. Setup floats: tables, (sub)figures, and captions
% ****************************************************************************************************
\usepackage{tabularx} % better tables
	\setlength{\extrarowheight}{3pt} % increase table row height
\newcommand{\tableheadline}[1]{\multicolumn{1}{c}{\spacedlowsmallcaps{#1}}}
\newcommand{\myfloatalign}{\centering} % to be used with each float for alignment
\usepackage{caption}
\captionsetup{format=hang,font=small}
\usepackage{subfig}  
% ****************************************************************************************************


% ****************************************************************************************************
% 5. Setup code listings
% ****************************************************************************************************
\usepackage{listings} 
%\lstset{emph={trueIndex,root},emphstyle=\color{BlueViolet}}%\underbar} % for special keywords
\lstset{language=[LaTeX]Tex,%C++,
    keywordstyle=\color{RoyalBlue},%\bfseries,
    basicstyle=\small\ttfamily,
    %identifierstyle=\color{NavyBlue},
    commentstyle=\color{Green}\ttfamily,
    stringstyle=\rmfamily,
    numbers=none,%left,%
    numberstyle=\scriptsize,%\tiny
    stepnumber=5,
    numbersep=8pt,
    showstringspaces=false,
    breaklines=true,
    frameround=ftff,
    frame=single,
    belowcaptionskip=.75\baselineskip
    %frame=L
} 
% ****************************************************************************************************    		   


% ****************************************************************************************************
% 6. PDFLaTeX, hyperreferences and citation backreferences
% ****************************************************************************************************
% ********************************************************************
% Using PDFLaTeX
% ********************************************************************
\PassOptionsToPackage{pdftex,hyperfootnotes=false,pdfpagelabels}{hyperref}
	\usepackage{hyperref}  % backref linktocpage pagebackref
\pdfcompresslevel=9
\pdfadjustspacing=1 
\PassOptionsToPackage{pdftex}{graphicx}
	\usepackage{graphicx} 

% ********************************************************************
% Setup the style of the backrefs from the bibliography
% (translate the options to any language you use)
% ********************************************************************
\newcommand{\backrefnotcitedstring}{\relax}%(Not cited.)
\newcommand{\backrefcitedsinglestring}[1]{(Cited on page~#1.)}
\newcommand{\backrefcitedmultistring}[1]{(Cited on pages~#1.)}
\ifthenelse{\boolean{enable-backrefs}}%
{%
		\PassOptionsToPackage{hyperpageref}{backref}
		\usepackage{backref} % to be loaded after hyperref package 
		   \renewcommand{\backreftwosep}{ and~} % separate 2 pages
		   \renewcommand{\backreflastsep}{, and~} % separate last of longer list
		   \renewcommand*{\backref}[1]{}  % disable standard
		   \renewcommand*{\backrefalt}[4]{% detailed backref
		      \ifcase #1 %
		         \backrefnotcitedstring%
		      \or%
		         \backrefcitedsinglestring{#2}%
		      \else%
		         \backrefcitedmultistring{#2}%
		      \fi}%
}{\relax}    

% ********************************************************************
% Hyperreferences
% ********************************************************************
\hypersetup{%
    %draft,	% = no hyperlinking at all (useful in b/w printouts)
    colorlinks=true, linktocpage=true, pdfstartpage=3, pdfstartview=FitV,%
    % uncomment the following line if you want to have black links (e.g., for printing)
    %colorlinks=false, linktocpage=false, pdfborder={0 0 0}, pdfstartpage=3, pdfstartview=FitV,% 
    breaklinks=true, pdfpagemode=UseNone, pageanchor=true, pdfpagemode=UseOutlines,%
    plainpages=false, bookmarksnumbered, bookmarksopen=true, bookmarksopenlevel=1,%
    hypertexnames=true, pdfhighlight=/O,%nesting=true,%frenchlinks,%
    urlcolor=webbrown, linkcolor=RoyalBlue, citecolor=webgreen, %pagecolor=RoyalBlue,%
    %urlcolor=Black, linkcolor=Black, citecolor=Black, %pagecolor=Black,%
    pdftitle={\myTitle},%
    pdfauthor={\textcopyright\ \myName, \myUni, \myFaculty},%
    pdfsubject={},%
    pdfkeywords={},%
    pdfcreator={pdfLaTeX},%
    pdfproducer={LaTeX with hyperref and classicthesis}%
}   

% ********************************************************************
% Setup autoreferences
% ********************************************************************
% There are some issues regarding autorefnames
% http://www.ureader.de/msg/136221647.aspx
% http://www.tex.ac.uk/cgi-bin/texfaq2html?label=latexwords
% you have to redefine the makros for the 
% language you use, e.g., american, ngerman
% (as chosen when loading babel/AtBeginDocument)
% ********************************************************************
\makeatletter
\@ifpackageloaded{babel}%
    {%
       \addto\extrasamerican{%
					\renewcommand*{\figureautorefname}{Figure}%
					\renewcommand*{\tableautorefname}{Table}%
					\renewcommand*{\partautorefname}{Part}%
					\renewcommand*{\chapterautorefname}{Chapter}%
					\renewcommand*{\sectionautorefname}{Section}%
					\renewcommand*{\subsectionautorefname}{Section}%
					\renewcommand*{\subsubsectionautorefname}{Section}% 	
				}%
       \addto\extrasngerman{% 
					\renewcommand*{\paragraphautorefname}{Absatz}%
					\renewcommand*{\subparagraphautorefname}{Unterabsatz}%
					\renewcommand*{\footnoteautorefname}{Fu\"snote}%
					\renewcommand*{\FancyVerbLineautorefname}{Zeile}%
					\renewcommand*{\theoremautorefname}{Theorem}%
					\renewcommand*{\appendixautorefname}{Anhang}%
					\renewcommand*{\equationautorefname}{Gleichung}%        
					\renewcommand*{\itemautorefname}{Punkt}%
				}%	
			% Fix to getting autorefs for subfigures right (thanks to Belinda Vogt for changing the definition)
			\providecommand{\subfigureautorefname}{\figureautorefname}%  			
    }{\relax}
\makeatother


% ****************************************************************************************************
% 7. Last calls before the bar closes
% ****************************************************************************************************
% ********************************************************************
% Development Stuff
% ********************************************************************
\listfiles
%\PassOptionsToPackage{l2tabu,orthodox,abort}{nag}
%	\usepackage{nag}
%\PassOptionsToPackage{warning, all}{onlyamsmath}
%	\usepackage{onlyamsmath}

% ********************************************************************
% Last, but not least...
% ********************************************************************
\usepackage{classicthesis} 
% ****************************************************************************************************


% ****************************************************************************************************
% 8. Further adjustments (experimental)
% ****************************************************************************************************
% ********************************************************************
% Changing the text area
% ********************************************************************
%\linespread{1.05} % a bit more for Palatino
%\areaset[current]{312pt}{761pt} % 686 (factor 2.2) + 33 head + 42 head \the\footskip
%\setlength{\marginparwidth}{7em}%
%\setlength{\marginparsep}{2em}%

% ********************************************************************
% Using different fonts
% ********************************************************************
%\usepackage[oldstylenums]{kpfonts} % oldstyle notextcomp
%\usepackage[osf]{libertine}
%\usepackage{hfoldsty} % Computer Modern with osf
%\usepackage[light,condensed,math]{iwona}
%\renewcommand{\sfdefault}{iwona}
%\usepackage{lmodern} % <-- no osf support :-(
%\usepackage[urw-garamond]{mathdesign} <-- no osf support :-(
% ****************************************************************************************************

% ****************************************************************************************************  
% If you like the classicthesis, then I would appreciate a postcard. 
% My address can be found in the file ClassicThesis.pdf. A collection 
% of the postcards I received so far is available online at 
% http://postcards.miede.de
% ****************************************************************************************************

% ****************************************************************************************************
% 1. Configure classicthesis for your needs here, e.g., remove "drafting" below 
% in order to deactivate the time-stamp on the pages
% ****************************************************************************************************
\PassOptionsToPackage{eulerchapternumbers,listings,drafting,%
				 pdfspacing,%floatperchapter,%linedheaders,%
				 subfig,beramono,eulermath,parts}{classicthesis}										
% ********************************************************************
% Available options for classicthesis.sty 
% (see ClassicThesis.pdf for more information):
% drafting
% parts nochapters linedheaders
% eulerchapternumbers beramono eulermath pdfspacing minionprospacing
% tocaligned dottedtoc manychapters
% listings floatperchapter subfig
% ********************************************************************

% ********************************************************************
% Triggers for this config
% ******************************************************************** 
\usepackage{ifthen}
\newboolean{enable-backrefs} % enable backrefs in the bibliography
\setboolean{enable-backrefs}{false} % true false
% ****************************************************************************************************


% ****************************************************************************************************
% 2. Personal data and user ad-hoc commands
% ****************************************************************************************************
\newcommand{\myTitle}{A Classic Thesis Style\xspace}
\newcommand{\mySubtitle}{An Homage to The Elements of Typographic Style\xspace}
\newcommand{\myDegree}{Doktor-Ingenieur (Dr.-Ing.)\xspace}
\newcommand{\myName}{Andr\'e Miede\xspace}
\newcommand{\myProf}{Put name here\xspace}
\newcommand{\myOtherProf}{Put name here\xspace}
\newcommand{\mySupervisor}{Put name here\xspace}
\newcommand{\myFaculty}{Put data here\xspace}
\newcommand{\myDepartment}{Put data here\xspace}
\newcommand{\myUni}{Put data here\xspace}
\newcommand{\myLocation}{Darmstadt\xspace}
\newcommand{\myTime}{December 2011\xspace}
\newcommand{\myVersion}{version 4.0\xspace}

% ********************************************************************
% Setup, finetuning, and useful commands
% ********************************************************************
\newcounter{dummy} % necessary for correct hyperlinks (to index, bib, etc.)
\newlength{\abcd} % for ab..z string length calculation
\providecommand{\mLyX}{L\kern-.1667em\lower.25em\hbox{Y}\kern-.125emX\@}
\newcommand{\ie}{i.\,e.}
\newcommand{\Ie}{I.\,e.}
\newcommand{\eg}{e.\,g.}
\newcommand{\Eg}{E.\,g.} 
% ****************************************************************************************************


% ****************************************************************************************************
% 3. Loading some handy packages
% ****************************************************************************************************
% ******************************************************************** 
% Packages with options that might require adjustments
% ******************************************************************** 
\PassOptionsToPackage{latin9}{inputenc}	% latin9 (ISO-8859-9) = latin1+"Euro sign"
 \usepackage{inputenc}				

%\PassOptionsToPackage{ngerman,american}{babel}   % change this to your language(s)
% Spanish languages need extra options in order to work with this template
%\PassOptionsToPackage{spanish,es-lcroman}{babel}
 \usepackage{babel}					

\PassOptionsToPackage{square,numbers}{natbib}
 \usepackage{natbib}				

\PassOptionsToPackage{fleqn}{amsmath}		% math environments and more by the AMS 
 \usepackage{amsmath}

% ******************************************************************** 
% General useful packages
% ******************************************************************** 
\PassOptionsToPackage{T1}{fontenc} % T2A for cyrillics
	\usepackage{fontenc}                 
\usepackage{xspace} % to get the spacing after macros right  
\usepackage{mparhack} % get marginpar right
\usepackage{fixltx2e} % fixes some LaTeX stuff 
\PassOptionsToPackage{printonlyused,smaller}{acronym}
	\usepackage{acronym} % nice macros for handling all acronyms in the thesis
%\renewcommand*{\acsfont}[1]{\textssc{#1}} % for MinionPro
\renewcommand{\bflabel}[1]{{#1}\hfill} % fix the list of acronyms
% ****************************************************************************************************


% ****************************************************************************************************
% 4. Setup floats: tables, (sub)figures, and captions
% ****************************************************************************************************
\usepackage{tabularx} % better tables
	\setlength{\extrarowheight}{3pt} % increase table row height
\newcommand{\tableheadline}[1]{\multicolumn{1}{c}{\spacedlowsmallcaps{#1}}}
\newcommand{\myfloatalign}{\centering} % to be used with each float for alignment
\usepackage{caption}
\captionsetup{format=hang,font=small}
\usepackage{subfig}  
% ****************************************************************************************************


% ****************************************************************************************************
% 5. Setup code listings
% ****************************************************************************************************
\usepackage{listings} 
%\lstset{emph={trueIndex,root},emphstyle=\color{BlueViolet}}%\underbar} % for special keywords
\lstset{language=[LaTeX]Tex,%C++,
    keywordstyle=\color{RoyalBlue},%\bfseries,
    basicstyle=\small\ttfamily,
    %identifierstyle=\color{NavyBlue},
    commentstyle=\color{Green}\ttfamily,
    stringstyle=\rmfamily,
    numbers=none,%left,%
    numberstyle=\scriptsize,%\tiny
    stepnumber=5,
    numbersep=8pt,
    showstringspaces=false,
    breaklines=true,
    frameround=ftff,
    frame=single,
    belowcaptionskip=.75\baselineskip
    %frame=L
} 
% ****************************************************************************************************    		   


% ****************************************************************************************************
% 6. PDFLaTeX, hyperreferences and citation backreferences
% ****************************************************************************************************
% ********************************************************************
% Using PDFLaTeX
% ********************************************************************
\PassOptionsToPackage{pdftex,hyperfootnotes=false,pdfpagelabels}{hyperref}
	\usepackage{hyperref}  % backref linktocpage pagebackref
\pdfcompresslevel=9
\pdfadjustspacing=1 
\PassOptionsToPackage{pdftex}{graphicx}
	\usepackage{graphicx} 

% ********************************************************************
% Setup the style of the backrefs from the bibliography
% (translate the options to any language you use)
% ********************************************************************
\newcommand{\backrefnotcitedstring}{\relax}%(Not cited.)
\newcommand{\backrefcitedsinglestring}[1]{(Cited on page~#1.)}
\newcommand{\backrefcitedmultistring}[1]{(Cited on pages~#1.)}
\ifthenelse{\boolean{enable-backrefs}}%
{%
		\PassOptionsToPackage{hyperpageref}{backref}
		\usepackage{backref} % to be loaded after hyperref package 
		   \renewcommand{\backreftwosep}{ and~} % separate 2 pages
		   \renewcommand{\backreflastsep}{, and~} % separate last of longer list
		   \renewcommand*{\backref}[1]{}  % disable standard
		   \renewcommand*{\backrefalt}[4]{% detailed backref
		      \ifcase #1 %
		         \backrefnotcitedstring%
		      \or%
		         \backrefcitedsinglestring{#2}%
		      \else%
		         \backrefcitedmultistring{#2}%
		      \fi}%
}{\relax}    

% ********************************************************************
% Hyperreferences
% ********************************************************************
\hypersetup{%
    %draft,	% = no hyperlinking at all (useful in b/w printouts)
    colorlinks=true, linktocpage=true, pdfstartpage=3, pdfstartview=FitV,%
    % uncomment the following line if you want to have black links (e.g., for printing)
    %colorlinks=false, linktocpage=false, pdfborder={0 0 0}, pdfstartpage=3, pdfstartview=FitV,% 
    breaklinks=true, pdfpagemode=UseNone, pageanchor=true, pdfpagemode=UseOutlines,%
    plainpages=false, bookmarksnumbered, bookmarksopen=true, bookmarksopenlevel=1,%
    hypertexnames=true, pdfhighlight=/O,%nesting=true,%frenchlinks,%
    urlcolor=webbrown, linkcolor=RoyalBlue, citecolor=webgreen, %pagecolor=RoyalBlue,%
    %urlcolor=Black, linkcolor=Black, citecolor=Black, %pagecolor=Black,%
    pdftitle={\myTitle},%
    pdfauthor={\textcopyright\ \myName, \myUni, \myFaculty},%
    pdfsubject={},%
    pdfkeywords={},%
    pdfcreator={pdfLaTeX},%
    pdfproducer={LaTeX with hyperref and classicthesis}%
}   

% ********************************************************************
% Setup autoreferences
% ********************************************************************
% There are some issues regarding autorefnames
% http://www.ureader.de/msg/136221647.aspx
% http://www.tex.ac.uk/cgi-bin/texfaq2html?label=latexwords
% you have to redefine the makros for the 
% language you use, e.g., american, ngerman
% (as chosen when loading babel/AtBeginDocument)
% ********************************************************************
\makeatletter
\@ifpackageloaded{babel}%
    {%
       \addto\extrasamerican{%
					\renewcommand*{\figureautorefname}{Figure}%
					\renewcommand*{\tableautorefname}{Table}%
					\renewcommand*{\partautorefname}{Part}%
					\renewcommand*{\chapterautorefname}{Chapter}%
					\renewcommand*{\sectionautorefname}{Section}%
					\renewcommand*{\subsectionautorefname}{Section}%
					\renewcommand*{\subsubsectionautorefname}{Section}% 	
				}%
       \addto\extrasngerman{% 
					\renewcommand*{\paragraphautorefname}{Absatz}%
					\renewcommand*{\subparagraphautorefname}{Unterabsatz}%
					\renewcommand*{\footnoteautorefname}{Fu\"snote}%
					\renewcommand*{\FancyVerbLineautorefname}{Zeile}%
					\renewcommand*{\theoremautorefname}{Theorem}%
					\renewcommand*{\appendixautorefname}{Anhang}%
					\renewcommand*{\equationautorefname}{Gleichung}%        
					\renewcommand*{\itemautorefname}{Punkt}%
				}%	
			% Fix to getting autorefs for subfigures right (thanks to Belinda Vogt for changing the definition)
			\providecommand{\subfigureautorefname}{\figureautorefname}%  			
    }{\relax}
\makeatother


% ****************************************************************************************************
% 7. Last calls before the bar closes
% ****************************************************************************************************
% ********************************************************************
% Development Stuff
% ********************************************************************
\listfiles
%\PassOptionsToPackage{l2tabu,orthodox,abort}{nag}
%	\usepackage{nag}
%\PassOptionsToPackage{warning, all}{onlyamsmath}
%	\usepackage{onlyamsmath}

% ********************************************************************
% Last, but not least...
% ********************************************************************
\usepackage{classicthesis} 
% ****************************************************************************************************


% ****************************************************************************************************
% 8. Further adjustments (experimental)
% ****************************************************************************************************
% ********************************************************************
% Changing the text area
% ********************************************************************
%\linespread{1.05} % a bit more for Palatino
%\areaset[current]{312pt}{761pt} % 686 (factor 2.2) + 33 head + 42 head \the\footskip
%\setlength{\marginparwidth}{7em}%
%\setlength{\marginparsep}{2em}%

% ********************************************************************
% Using different fonts
% ********************************************************************
%\usepackage[oldstylenums]{kpfonts} % oldstyle notextcomp
%\usepackage[osf]{libertine}
%\usepackage{hfoldsty} % Computer Modern with osf
%\usepackage[light,condensed,math]{iwona}
%\renewcommand{\sfdefault}{iwona}
%\usepackage{lmodern} % <-- no osf support :-(
%\usepackage[urw-garamond]{mathdesign} <-- no osf support :-(
% ****************************************************************************************************

% ****************************************************************************************************  
% If you like the classicthesis, then I would appreciate a postcard. 
% My address can be found in the file ClassicThesis.pdf. A collection 
% of the postcards I received so far is available online at 
% http://postcards.miede.de
% ****************************************************************************************************

% ****************************************************************************************************
% 1. Configure classicthesis for your needs here, e.g., remove "drafting" below 
% in order to deactivate the time-stamp on the pages
% ****************************************************************************************************
\PassOptionsToPackage{eulerchapternumbers,listings,drafting,%
				 pdfspacing,%floatperchapter,%linedheaders,%
				 subfig,beramono,eulermath,parts}{classicthesis}										
% ********************************************************************
% Available options for classicthesis.sty 
% (see ClassicThesis.pdf for more information):
% drafting
% parts nochapters linedheaders
% eulerchapternumbers beramono eulermath pdfspacing minionprospacing
% tocaligned dottedtoc manychapters
% listings floatperchapter subfig
% ********************************************************************

% ********************************************************************
% Triggers for this config
% ******************************************************************** 
\usepackage{ifthen}
\newboolean{enable-backrefs} % enable backrefs in the bibliography
\setboolean{enable-backrefs}{false} % true false
% ****************************************************************************************************


% ****************************************************************************************************
% 2. Personal data and user ad-hoc commands
% ****************************************************************************************************
\newcommand{\myTitle}{A Classic Thesis Style\xspace}
\newcommand{\mySubtitle}{An Homage to The Elements of Typographic Style\xspace}
\newcommand{\myDegree}{Doktor-Ingenieur (Dr.-Ing.)\xspace}
\newcommand{\myName}{Andr\'e Miede\xspace}
\newcommand{\myProf}{Put name here\xspace}
\newcommand{\myOtherProf}{Put name here\xspace}
\newcommand{\mySupervisor}{Put name here\xspace}
\newcommand{\myFaculty}{Put data here\xspace}
\newcommand{\myDepartment}{Put data here\xspace}
\newcommand{\myUni}{Put data here\xspace}
\newcommand{\myLocation}{Darmstadt\xspace}
\newcommand{\myTime}{December 2011\xspace}
\newcommand{\myVersion}{version 4.0\xspace}

% ********************************************************************
% Setup, finetuning, and useful commands
% ********************************************************************
\newcounter{dummy} % necessary for correct hyperlinks (to index, bib, etc.)
\newlength{\abcd} % for ab..z string length calculation
\providecommand{\mLyX}{L\kern-.1667em\lower.25em\hbox{Y}\kern-.125emX\@}
\newcommand{\ie}{i.\,e.}
\newcommand{\Ie}{I.\,e.}
\newcommand{\eg}{e.\,g.}
\newcommand{\Eg}{E.\,g.} 
% ****************************************************************************************************


% ****************************************************************************************************
% 3. Loading some handy packages
% ****************************************************************************************************
% ******************************************************************** 
% Packages with options that might require adjustments
% ******************************************************************** 
\PassOptionsToPackage{latin9}{inputenc}	% latin9 (ISO-8859-9) = latin1+"Euro sign"
 \usepackage{inputenc}				

%\PassOptionsToPackage{ngerman,american}{babel}   % change this to your language(s)
% Spanish languages need extra options in order to work with this template
%\PassOptionsToPackage{spanish,es-lcroman}{babel}
 \usepackage{babel}					

\PassOptionsToPackage{square,numbers}{natbib}
 \usepackage{natbib}				

\PassOptionsToPackage{fleqn}{amsmath}		% math environments and more by the AMS 
 \usepackage{amsmath}

% ******************************************************************** 
% General useful packages
% ******************************************************************** 
\PassOptionsToPackage{T1}{fontenc} % T2A for cyrillics
	\usepackage{fontenc}                 
\usepackage{xspace} % to get the spacing after macros right  
\usepackage{mparhack} % get marginpar right
\usepackage{fixltx2e} % fixes some LaTeX stuff 
\PassOptionsToPackage{printonlyused,smaller}{acronym}
	\usepackage{acronym} % nice macros for handling all acronyms in the thesis
%\renewcommand*{\acsfont}[1]{\textssc{#1}} % for MinionPro
\renewcommand{\bflabel}[1]{{#1}\hfill} % fix the list of acronyms
% ****************************************************************************************************


% ****************************************************************************************************
% 4. Setup floats: tables, (sub)figures, and captions
% ****************************************************************************************************
\usepackage{tabularx} % better tables
	\setlength{\extrarowheight}{3pt} % increase table row height
\newcommand{\tableheadline}[1]{\multicolumn{1}{c}{\spacedlowsmallcaps{#1}}}
\newcommand{\myfloatalign}{\centering} % to be used with each float for alignment
\usepackage{caption}
\captionsetup{format=hang,font=small}
\usepackage{subfig}  
% ****************************************************************************************************


% ****************************************************************************************************
% 5. Setup code listings
% ****************************************************************************************************
\usepackage{listings} 
%\lstset{emph={trueIndex,root},emphstyle=\color{BlueViolet}}%\underbar} % for special keywords
\lstset{language=[LaTeX]Tex,%C++,
    keywordstyle=\color{RoyalBlue},%\bfseries,
    basicstyle=\small\ttfamily,
    %identifierstyle=\color{NavyBlue},
    commentstyle=\color{Green}\ttfamily,
    stringstyle=\rmfamily,
    numbers=none,%left,%
    numberstyle=\scriptsize,%\tiny
    stepnumber=5,
    numbersep=8pt,
    showstringspaces=false,
    breaklines=true,
    frameround=ftff,
    frame=single,
    belowcaptionskip=.75\baselineskip
    %frame=L
} 
% ****************************************************************************************************    		   


% ****************************************************************************************************
% 6. PDFLaTeX, hyperreferences and citation backreferences
% ****************************************************************************************************
% ********************************************************************
% Using PDFLaTeX
% ********************************************************************
\PassOptionsToPackage{pdftex,hyperfootnotes=false,pdfpagelabels}{hyperref}
	\usepackage{hyperref}  % backref linktocpage pagebackref
\pdfcompresslevel=9
\pdfadjustspacing=1 
\PassOptionsToPackage{pdftex}{graphicx}
	\usepackage{graphicx} 

% ********************************************************************
% Setup the style of the backrefs from the bibliography
% (translate the options to any language you use)
% ********************************************************************
\newcommand{\backrefnotcitedstring}{\relax}%(Not cited.)
\newcommand{\backrefcitedsinglestring}[1]{(Cited on page~#1.)}
\newcommand{\backrefcitedmultistring}[1]{(Cited on pages~#1.)}
\ifthenelse{\boolean{enable-backrefs}}%
{%
		\PassOptionsToPackage{hyperpageref}{backref}
		\usepackage{backref} % to be loaded after hyperref package 
		   \renewcommand{\backreftwosep}{ and~} % separate 2 pages
		   \renewcommand{\backreflastsep}{, and~} % separate last of longer list
		   \renewcommand*{\backref}[1]{}  % disable standard
		   \renewcommand*{\backrefalt}[4]{% detailed backref
		      \ifcase #1 %
		         \backrefnotcitedstring%
		      \or%
		         \backrefcitedsinglestring{#2}%
		      \else%
		         \backrefcitedmultistring{#2}%
		      \fi}%
}{\relax}    

% ********************************************************************
% Hyperreferences
% ********************************************************************
\hypersetup{%
    %draft,	% = no hyperlinking at all (useful in b/w printouts)
    colorlinks=true, linktocpage=true, pdfstartpage=3, pdfstartview=FitV,%
    % uncomment the following line if you want to have black links (e.g., for printing)
    %colorlinks=false, linktocpage=false, pdfborder={0 0 0}, pdfstartpage=3, pdfstartview=FitV,% 
    breaklinks=true, pdfpagemode=UseNone, pageanchor=true, pdfpagemode=UseOutlines,%
    plainpages=false, bookmarksnumbered, bookmarksopen=true, bookmarksopenlevel=1,%
    hypertexnames=true, pdfhighlight=/O,%nesting=true,%frenchlinks,%
    urlcolor=webbrown, linkcolor=RoyalBlue, citecolor=webgreen, %pagecolor=RoyalBlue,%
    %urlcolor=Black, linkcolor=Black, citecolor=Black, %pagecolor=Black,%
    pdftitle={\myTitle},%
    pdfauthor={\textcopyright\ \myName, \myUni, \myFaculty},%
    pdfsubject={},%
    pdfkeywords={},%
    pdfcreator={pdfLaTeX},%
    pdfproducer={LaTeX with hyperref and classicthesis}%
}   

% ********************************************************************
% Setup autoreferences
% ********************************************************************
% There are some issues regarding autorefnames
% http://www.ureader.de/msg/136221647.aspx
% http://www.tex.ac.uk/cgi-bin/texfaq2html?label=latexwords
% you have to redefine the makros for the 
% language you use, e.g., american, ngerman
% (as chosen when loading babel/AtBeginDocument)
% ********************************************************************
\makeatletter
\@ifpackageloaded{babel}%
    {%
       \addto\extrasamerican{%
					\renewcommand*{\figureautorefname}{Figure}%
					\renewcommand*{\tableautorefname}{Table}%
					\renewcommand*{\partautorefname}{Part}%
					\renewcommand*{\chapterautorefname}{Chapter}%
					\renewcommand*{\sectionautorefname}{Section}%
					\renewcommand*{\subsectionautorefname}{Section}%
					\renewcommand*{\subsubsectionautorefname}{Section}% 	
				}%
       \addto\extrasngerman{% 
					\renewcommand*{\paragraphautorefname}{Absatz}%
					\renewcommand*{\subparagraphautorefname}{Unterabsatz}%
					\renewcommand*{\footnoteautorefname}{Fu\"snote}%
					\renewcommand*{\FancyVerbLineautorefname}{Zeile}%
					\renewcommand*{\theoremautorefname}{Theorem}%
					\renewcommand*{\appendixautorefname}{Anhang}%
					\renewcommand*{\equationautorefname}{Gleichung}%        
					\renewcommand*{\itemautorefname}{Punkt}%
				}%	
			% Fix to getting autorefs for subfigures right (thanks to Belinda Vogt for changing the definition)
			\providecommand{\subfigureautorefname}{\figureautorefname}%  			
    }{\relax}
\makeatother


% ****************************************************************************************************
% 7. Last calls before the bar closes
% ****************************************************************************************************
% ********************************************************************
% Development Stuff
% ********************************************************************
\listfiles
%\PassOptionsToPackage{l2tabu,orthodox,abort}{nag}
%	\usepackage{nag}
%\PassOptionsToPackage{warning, all}{onlyamsmath}
%	\usepackage{onlyamsmath}

% ********************************************************************
% Last, but not least...
% ********************************************************************
\usepackage{classicthesis} 
% ****************************************************************************************************


% ****************************************************************************************************
% 8. Further adjustments (experimental)
% ****************************************************************************************************
% ********************************************************************
% Changing the text area
% ********************************************************************
%\linespread{1.05} % a bit more for Palatino
%\areaset[current]{312pt}{761pt} % 686 (factor 2.2) + 33 head + 42 head \the\footskip
%\setlength{\marginparwidth}{7em}%
%\setlength{\marginparsep}{2em}%

% ********************************************************************
% Using different fonts
% ********************************************************************
%\usepackage[oldstylenums]{kpfonts} % oldstyle notextcomp
%\usepackage[osf]{libertine}
%\usepackage{hfoldsty} % Computer Modern with osf
%\usepackage[light,condensed,math]{iwona}
%\renewcommand{\sfdefault}{iwona}
%\usepackage{lmodern} % <-- no osf support :-(
%\usepackage[urw-garamond]{mathdesign} <-- no osf support :-(
% ****************************************************************************************************

% ****************************************************************************************************  
% If you like the classicthesis, then I would appreciate a postcard. 
% My address can be found in the file ClassicThesis.pdf. A collection 
% of the postcards I received so far is available online at 
% http://postcards.miede.de
% ****************************************************************************************************

% ****************************************************************************************************
% 1. Configure classicthesis for your needs here, e.g., remove "drafting" below 
% in order to deactivate the time-stamp on the pages
% ****************************************************************************************************
\PassOptionsToPackage{eulerchapternumbers,listings,drafting,%
				 pdfspacing,%floatperchapter,%linedheaders,%
				 subfig,beramono,minionpro,minionprospacing,dottedtoc}{classicthesis}										
% ********************************************************************
% Available options for classicthesis.sty 
% (see ClassicThesis.pdf for more information):
% drafting
% parts nochapters linedheaders
% eulerchapternumbers beramono eulermath pdfspacing minionprospacing
% tocaligned dottedtoc manychapters
% listings floatperchapter subfig
% ********************************************************************

% ********************************************************************
% Triggers for this config
% ******************************************************************** 
\usepackage{ifthen}
\newboolean{enable-backrefs} % enable backrefs in the bibliography
\setboolean{enable-backrefs}{false} % true false
% ****************************************************************************************************


% ****************************************************************************************************
% 2. Personal data and user ad-hoc commands
% ****************************************************************************************************
\newcommand{\myTitle}{A Classic Thesis Style\xspace}
\newcommand{\mySubtitle}{An Homage to The Elements of Typographic Style\xspace}
\newcommand{\myDegree}{Master Thesis\xspace}
\newcommand{\myName}{Andr\'e Miede\xspace}
\newcommand{\myProf}{Put name here\xspace}
\newcommand{\myOtherProf}{Put name here\xspace}
\newcommand{\mySupervisor}{Put name here\xspace}
\newcommand{\myOtherSupervisor}{Put name here\xspace}
\newcommand{\myFaculty}{Put data here\xspace}
\newcommand{\myUni}{Put data here\xspace}
\newcommand{\myLocation}{Darmstadt\xspace}
\newcommand{\myTime}{December 2011\xspace}
\newcommand{\myVersion}{version 4.0\xspace}

% ********************************************************************
% Setup, finetuning, and useful commands
% ********************************************************************
\newcounter{dummy} % necessary for correct hyperlinks (to index, bib, etc.)
\newlength{\abcd} % for ab..z string length calculation
\providecommand{\mLyX}{L\kern-.1667em\lower.25em\hbox{Y}\kern-.125emX\@}
\newcommand{\ie}{i.\,e.}
\newcommand{\Ie}{I.\,e.}
\newcommand{\eg}{e.\,g.}
\newcommand{\Eg}{E.\,g.} 
% ****************************************************************************************************


% ****************************************************************************************************
% 3. Loading some handy packages
% ****************************************************************************************************
% ******************************************************************** 
% Packages with options that might require adjustments
% ******************************************************************** 
\PassOptionsToPackage{utf8}{inputenc}	% latin9 (ISO-8859-9) = latin1+"Euro sign"
 \usepackage{inputenc}				

\PassOptionsToPackage{american}{babel}   % change this to your language(s)
% Spanish languages need extra options in order to work with this template
%\PassOptionsToPackage{spanish,es-lcroman}{babel}
 \usepackage{babel}
 
\PassOptionsToPackage{german=guillemets}{csquotes}
 \usepackage{csquotes}

%\PassOptionsToPackage{square,numbers}{natbib}
% \usepackage{natbib}	

\PassOptionsToPackage{%
    style=authoryear-comp,%
    bibstyle=authoryear,
    %ibidtracker=strict,
    %opcittracker=strict,
    %loccittracker=strict,
    sortcites=true,%
    sorting=nyt,%
    block=space,%
    backref=true,%
    labelnumber=true,
    maxcitenames=1,
    uniquename=full,
    uniquelist=false,
    backend=biber%
}{biblatex}
\usepackage{biblatex}
\DefineBibliographyExtras{american}{\stdpunctuation} % logical punctuation
\setlength{\bibitemsep}{1ex}
%\setlength{\bibhang}{0pt}
\bibliography{Bibliography}

% suppress et al for text citations (only first author and year)
\renewbibmacro*{name:andothers}{} 
\AtBeginBibliography{% 
\renewbibmacro*{name:andothers}{% 
\ifthenelse{\value{listcount}=\value{liststop}\AND 
\ifmorenames} 
{\ifnum\value{liststop}>1 \finalandcomma\fi 
\andothersdelim\bibstring{andothers}} 
{}}}


\PassOptionsToPackage{fleqn}{amsmath}		% math environments and more by the AMS 
 \usepackage{amsmath}

\usepackage{mathabx} % for the dlsh symbol.


% ******************************************************************** 
% General useful packages
% ******************************************************************** 
\PassOptionsToPackage{T1}{fontenc} % T2A for cyrillics
	\usepackage{fontenc}            
	\usepackage{textcomp}     
\usepackage{xspace} % to get the spacing after macros right  
\usepackage{mparhack} % get marginpar right
\usepackage{fixltx2e} % fixes some LaTeX stuff 
\PassOptionsToPackage{printonlyused}{acronym} %smaller
	\usepackage{acronym} % nice macros for handling all acronyms in the thesis
%\renewcommand*{\acsfont}[1]{\textssc{#1}} % for MinionPro

%% used for having spaced small caps as acronyms
% \makeatletter 
% \renewcommand*{\AC@hyperlink}[2]{% 
% \hyper@@link{}{#1}{\textssc{\lowercase{#2}}}} 
% \renewcommand*{\AC@acro}[1]{% 
% \@ifnextchar[{\AC@@acro{#1}}{\AC@@acro{#1}[\textssc{\lowercase{#1}}]}} 
% \makeatother

\newcommand{\nara}[1]{#1} % = not a real acronym / if you change the styling of the \acro acronyms
                          % make sure that all your other acronyms can be changed in one place as
                          % well
                          
\renewcommand{\bflabel}[1]{{#1}\hfill} % fix the list of acronyms

\usepackage{numprint} % pretty print numbers with decimal and thousands sep
\usepackage{framed}

\usepackage{layouts}
\usepackage{longtable}
\usepackage{multirow}
\usepackage{titletoc}

\usepackage{highlighter}
\usepackage{relsize} % scale font up or down

% ****************************************************************************************************



% ********************************************************************
% Commands
% ********************************************************************
\newcounter{dummy} % necessary for correct hyperlinks (to index, bib, etc.)
\newcommand{\longpage}{\enlargethispage{3\baselineskip}}
\newcommand{\medpage}{\enlargethispage{2\baselineskip}}
\newcommand{\shortpage}{\enlargethispage{\baselineskip}}
\newcommand{\concskip}{\medskip}
\newcommand{\myPrimRef}[1]{\textit{#1}} % how primary index entries will be marked
% ********************************************************************
\newcommand{\ie}{i.\,e.}
\newcommand{\Ie}{I.\,e.}
\newcommand{\eg}{e.\,g.}
\newcommand{\Eg}{E.\,g.}
\newcommand{\zb}{z.\,B.}
\newcommand{\etAl}{et al.\xspace}
\newcommand{\twod}{2\textsc{d}\xspace}
\newcommand{\threed}{3\textsc{d}\xspace}
% ********************************************************************
\newcommand{\system}[1]{\emph{#1}}
\newcommand{\person}[1]{#1}
\newcommand{\httpmethod}[1]{#1}
\newcommand{\term}[1]{\emph{#1}}
\newcommand{\role}[1]{#1} %textsc
\newcommand{\cauthor}[1]{\relax}
\newcommand{\tableheadline}[1]{\multicolumn{1}{c}{\spacedlowsmallcaps{#1}}}
% ********************************************************************
\newcommand{\RA}{\ensuremath{\Rightarrow}}
\newcommand{\ra}{\ensuremath{\rightarrow}}
\newcommand{\lra}{\ensuremath{\leftrightarrow}}
\newcommand{\On}{\ensuremath{O(n)}\xspace}
\newcommand{\Ologn}{\ensuremath{O(\log n)}\xspace}
\newcommand{\Oone}{\ensuremath{O(1)}\xspace}



% ****************************************************************************************************
% 4. Setup floats: tables, (sub)figures, and captions
% ****************************************************************************************************
\usepackage{tabularx} % better tables
	\setlength{\extrarowheight}{3pt} % increase table row height
\newcommand{\tableheadline}[1]{\multicolumn{1}{c}{\spacedlowsmallcaps{#1}}}
\newcommand{\myfloatalign}{\centering} % to be used with each float for alignment
\usepackage{caption}
\usepackage{subfig} 

%\captionsetup{format=hang,font=small}
\DeclareCaptionLabelFormat{slsc}{\spacedlowsmallcaps{#1 #2}}
\DeclareCaptionLabelSeparator{spacednewline}{\\[1ex]}
\captionsetup[figure]{format=plain,labelsep=colon,labelfont=it,textfont=rm,font=small,singlelinecheck=true,position=bottom}
\captionsetup[table]{format=plain,labelsep=colon,labelfont=it,textfont=rm,font=small,singlelinecheck=true,position=top}
\captionsetup[SCfigure]{format=plain,labelsep=spacednewline,labelfont={rm,small},labelformat=slsc,textfont=rm,font=footnotesize,singlelinecheck=true,position=top}
\captionsetup[SCtable]{format=plain,labelsep=spacednewline,labelfont={rm,small},labelformat=slsc,textfont=rm,font=footnotesize,singlelinecheck=true,position=top}


% ****************************************************************************************************


% ****************************************************************************************************
% 5. Setup code listings
% ****************************************************************************************************
\usepackage{listings} 
%\lstset{emph={trueIndex,root},emphstyle=\color{BlueViolet}}%\underbar} % for special keywords
\lstset{language=[LaTeX]Tex,%C++,
    keywordstyle=\color{RoyalBlue},%\bfseries,
    basicstyle=\small\ttfamily,
    %identifierstyle=\color{NavyBlue},
    commentstyle=\color{Green}\ttfamily,
    stringstyle=\rmfamily,
    numbers=none,%left,%
    numberstyle=\scriptsize,%\tiny
    stepnumber=5,
    numbersep=8pt,
    showstringspaces=false,
    breaklines=true,
    frameround=ftff,
    frame=single,
    belowcaptionskip=.75\baselineskip
    %frame=L
} 
% ****************************************************************************************************    		   


% ****************************************************************************************************
% 6. PDFLaTeX, hyperreferences and citation backreferences
% ****************************************************************************************************
% ********************************************************************
% Using PDFLaTeX
% ********************************************************************
\PassOptionsToPackage{pdftex,hyperfootnotes=false,pdfpagelabels}{hyperref}
	\usepackage{hyperref}  % backref linktocpage pagebackref
\pdfcompresslevel=9
\pdfadjustspacing=1 
\PassOptionsToPackage{pdftex}{graphicx}
	\usepackage{graphicx} 

% ********************************************************************
% Setup the style of the backrefs from the bibliography
% (translate the options to any language you use)
% ********************************************************************
\newcommand{\backrefnotcitedstring}{\relax}%(Not cited.)
\newcommand{\backrefcitedsinglestring}[1]{(Cited on page~#1.)}
\newcommand{\backrefcitedmultistring}[1]{(Cited on pages~#1.)}
\ifthenelse{\boolean{enable-backrefs}}%
{%
		\PassOptionsToPackage{hyperpageref}{backref}
		\usepackage{backref} % to be loaded after hyperref package 
		   \renewcommand{\backreftwosep}{ and~} % separate 2 pages
		   \renewcommand{\backreflastsep}{, and~} % separate last of longer list
		   \renewcommand*{\backref}[1]{}  % disable standard
		   \renewcommand*{\backrefalt}[4]{% detailed backref
		      \ifcase #1 %
		         \backrefnotcitedstring%
		      \or%
		         \backrefcitedsinglestring{#2}%
		      \else%
		         \backrefcitedmultistring{#2}%
		      \fi}%
}{\relax}    

% ********************************************************************
% Hyperreferences
% ********************************************************************
\hypersetup{%
    %draft,	% = no hyperlinking at all (useful in b/w printouts)
    colorlinks=false, linktocpage=false, pdfborder={0 0 0}, pdfstartpage=3, pdfstartview=FitV,%
    % uncomment the following line if you want to have black links (e.g., for printing)
    %colorlinks=false, linktocpage=false, pdfborder={0 0 0}, pdfstartpage=3, pdfstartview=FitV,% 
    breaklinks=true, pdfpagemode=UseNone, pageanchor=true, pdfpagemode=UseOutlines,%
    plainpages=false, bookmarksnumbered, bookmarksopen=true, bookmarksopenlevel=1,%
    hypertexnames=true, pdfhighlight=/O,%nesting=true,%frenchlinks,%
    urlcolor=webbrown, linkcolor=RoyalBlue, citecolor=webgreen, %pagecolor=RoyalBlue,%
    %urlcolor=Black, linkcolor=Black, citecolor=Black, %pagecolor=Black,%
    pdftitle={\myTitle},%
    pdfauthor={\textcopyright\ \myName, \myUni, \myFaculty},%
    pdfsubject={},%
    pdfkeywords={},%
    pdfcreator={pdfLaTeX},%
    pdfproducer={LaTeX with hyperref and classicthesis}%
}   


% no footnote resetting on chapter break
\RequirePackage{chngcntr}
\counterwithout*{footnote}{chapter}



% my own commands
\newcommand{\ed}[1]{\textbf{\color{red}#1}}
\newcommand{\cn}{\ed{[CITATION NEEDED]}}
\newcommand{\furl}[2]{\href{#2}{#1}\footnote{\url{#2}}}
\newcommand{\slsc}[1]{\spacedlowsmallcaps{#1}}
\newcommand{\sac}[1]{\spacedallcaps{#1}}
\newcommand{\wpcite}[3]{\footnote{Quoted from \emph{Wikipedia, The Free Encyclopedia}, Lemma \enquote{#1}.\\\url{http://en.wikipedia.org/w/index.php?title=#2&oldid=#3}, accessed on January 15, 2012.}}
\newcommand{\see}[1]{\marginnote{see \S~\ref{#1}}}
\newcommand{\ssee}[1]{\marginnote{see \S\S~\ref{#1}}}

\newcommand{\marginnote}[1]{\mbox{}\marginpar{#1}}
\newcommand{\edm}[1]{\marginnote{\ed{#1}}}

\renewcommand{\thefootnote}{\figureversion{osf,tab}\arabic{footnote}}


% The placeins package provides the \FloatBarrier command.  This forces
% LaTeX to place all of the floats before proceeding.  We'll use this to
% keep the float (figure and table) numbers in sequence.
\RequirePackage{placeins}

%%
% Margin float environment

\newsavebox{\@tufte@margin@floatbox}
\newenvironment{@tufte@margin@float}[2][-1.2ex]%
  {\FloatBarrier% process all floats before this point so the figure/table numbers stay in order.
  \begin{lrbox}{\@tufte@margin@floatbox}%
  \begin{minipage}{\marginparwidth}%
    \hbox{}\vspace*{#1}%
    \captionsetup{type=#2}%
    \def\@captype{#2}%
    %\@tufte@marginfont%
    %\@tufte@justification%
    %\@tufte@margin@par%
    \noindent\ignorespaces%
  }
  {\end{minipage}%
  \end{lrbox}%
  \marginpar{\usebox{\@tufte@margin@floatbox}}%
  }


%%
% Margin figure environment

\newenvironment{marginfigure}[1][-1.2ex]%
  {\begin{@tufte@margin@float}[#1]{figure}}
  {\end{@tufte@margin@float}}

\newenvironment{margintable}[1][-1.2ex]%
    {\begin{@tufte@margin@float}[#1]{table}}
    {\end{@tufte@margin@float}}


\PassOptionsToPackage{margincaption,outercaption,ragged,wide}{sidecap}
 \RequirePackage{sidecap}
 \sidecaptionvpos{figure}{t}
 \sidecaptionvpos{table}{t}



% ********************************************************************
% Setup autoreferences
% ********************************************************************
% There are some issues regarding autorefnames
% http://www.ureader.de/msg/136221647.aspx
% http://www.tex.ac.uk/cgi-bin/texfaq2html?label=latexwords
% you have to redefine the makros for the 
% language you use, e.g., american, ngerman
% (as chosen when loading babel/AtBeginDocument)
% ********************************************************************
\makeatletter
\@ifpackageloaded{babel}%
    {%
       \addto\extrasamerican{%
					\renewcommand*{\figureautorefname}{Figure}%
					\renewcommand*{\tableautorefname}{Table}%
					\renewcommand*{\partautorefname}{Part}%
					\renewcommand*{\chapterautorefname}{\S}%
					\renewcommand*{\sectionautorefname}{\S}%
					\renewcommand*{\subsectionautorefname}{\S}%
					\renewcommand*{\subsubsectionautorefname}{\S}% 	
				}%
        %        \addto\extrasngerman{% 
        %   \renewcommand*{\paragraphautorefname}{Absatz}%
        %   \renewcommand*{\subparagraphautorefname}{Unterabsatz}%
        %   \renewcommand*{\footnoteautorefname}{Fu\"snote}%
        %   \renewcommand*{\FancyVerbLineautorefname}{Zeile}%
        %   \renewcommand*{\theoremautorefname}{Theorem}%
        %   \renewcommand*{\appendixautorefname}{Anhang}%
        %   \renewcommand*{\equationautorefname}{Gleichung}%        
        %   \renewcommand*{\itemautorefname}{Punkt}%
        % }%  
			% Fix to getting autorefs for subfigures right (thanks to Belinda Vogt for changing the definition)
			\providecommand{\subfigureautorefname}{\figureautorefname}%  			
    }{\relax}
\makeatother


% ****************************************************************************************************
% 7. Last calls before the bar closes
% ****************************************************************************************************
% ********************************************************************
% Development Stuff
% ********************************************************************
\listfiles
%\PassOptionsToPackage{l2tabu,orthodox,abort}{nag}
%	\usepackage{nag}
%\PassOptionsToPackage{warning, all}{onlyamsmath}
%	\usepackage{onlyamsmath}

% ********************************************************************
% Last, but not least...
% ********************************************************************
\usepackage{classicthesis}

\edef\sidecaptionsep{\the\marginparsep}
% ****************************************************************************************************


% ****************************************************************************************************
% 8. Further adjustments (experimental)
% ****************************************************************************************************
% ********************************************************************
% Changing the text area
% ********************************************************************
%\linespread{1.05} % a bit more for Palatino
%\areaset[current]{312pt}{761pt} % 686 (factor 2.2) + 33 head + 42 head \the\footskip
%\setlength{\marginparwidth}{7em}%
%\setlength{\marginparsep}{2em}%

% ********************************************************************
% Using different fonts
% ********************************************************************
%\usepackage[oldstylenums]{kpfonts} % oldstyle notextcomp
%\usepackage[osf]{libertine}
%\usepackage{hfoldsty} % Computer Modern with osf
%\usepackage[light,condensed,math]{iwona}
%\renewcommand{\sfdefault}{iwona}
%\usepackage{lmodern} % <-- no osf support :-(
%\usepackage[urw-garamond]{mathdesign} <-- no osf support :-(
% ****************************************************************************************************
